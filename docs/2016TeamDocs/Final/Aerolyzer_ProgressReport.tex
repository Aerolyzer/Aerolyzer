\documentclass[onecolumn, draftclsnofoot,10pt, compsoc]{IEEEtran}
\usepackage{graphicx}
\usepackage{url}
\usepackage{setspace}
\usepackage[margin=0.75in]{geometry}
\setlength{\parindent}{0pt}
\usepackage[hidelinks]{hyperref}
\usepackage{listings}
\usepackage{float}

\renewcommand\thesection{\Roman{section}}
\renewcommand\thesubsection{\Alph{subsection}}
\renewcommand\thesubsubsection{\arabic{subsubsection}}
\geometry{textheight=9.5in, textwidth=7in}

% 1. Fill in these details
\def \CapstoneTeamName{     Aerolyzer}
\def \CapstoneTeamNumber{       22}
\def \GroupMemberOne{           E. Reilly Collins}
\def \GroupMemberTwo{           Sophia Liu}
\def \GroupMemberThree{         Jesse Hanson}
\def \CapstoneProjectName{      Aerosol Analyzer Mobile Web Application}
\def \CapstoneSponsorCompany{   NASA JPL}
\def \CapstoneSponsorPerson{        Kim Whitehall, Lewis McGibbney}

% 2. Uncomment the appropriate line below so that the document type works
\def \DocType{		%Problem Statement
				%Requirements Document
				%Technology Review
				%Design Document
				Progress Report
				}
			
\newcommand{\NameSigPair}[1]{\par
\makebox[2.75in][r]{#1} \hfil 	\makebox[3.25in]{\makebox[2.25in]{\hrulefill} \hfill		\makebox[.75in]{\hrulefill}}
\par\vspace{-12pt} \textit{\tiny\noindent
\makebox[2.75in]{} \hfil		\makebox[3.25in]{\makebox[2.25in][r]{Signature} \hfill	\makebox[.75in][r]{Date}}}}
% 3. If the document is not to be signed, uncomment the RENEWcommand below
\renewcommand{\NameSigPair}[1]{#1}

%%%%%%%%%%%%%%%%%%%%%%%%%%%%%%%%%%%%%%%
\begin{document}
\begin{titlepage}
    \pagenumbering{gobble}
    \begin{singlespace}
    	%\includegraphics[height=4cm]{coe_v_spot1}
        \hfill 
        % 4. If you have a logo, use this includegraphics command to put it on the coversheet.
        %\includegraphics[height=4cm]{CompanyLogo}   
        \par\vspace{.2in}
        \centering
        \scshape{
            \huge CS Capstone \DocType \par
            {\large\today}\par
            \vspace{.5in}
            \textbf{\Huge\CapstoneProjectName}\par
            \vfill
            {\large Prepared for}\par
            \Huge \CapstoneSponsorCompany\par
            \vspace{5pt}
            {\Large\NameSigPair{\CapstoneSponsorPerson}\par}
            {\large Prepared by }\par
            Group\CapstoneTeamNumber\par
            % 5. comment out the line below this one if you do not wish to name your team
            \CapstoneTeamName\par 
            \vspace{5pt}
            {\Large
                \NameSigPair{\GroupMemberOne}\par
                \NameSigPair{\GroupMemberTwo}\par
                \NameSigPair{\GroupMemberThree}\par
            }
            \vspace{18pt}
        }
        \begin{abstract}
        % 6. Fill in your abstract   
        \noindent Over the past 10 weeks, the Aerolyzer team has worked with our client to make progress towards our finished software product. The purpose of this document is to illustrate that progress and provide a retrospective of our work done so far. A summary the purpose and goals of our project, our current status, and our weekly progress is offered. Additionally, we explain the positives, things that will change, and specific actions to be taken for these changes to occur. We are on track to have our aerosol-analyzing mobile application ready in time for the Expo.

        \medskip

        In summary, we have now completed several documents that aided us in figuring out how our project will come together. Furthermore, we have worked on several tasks assigned by our client and added these to our code repository.  Lastly, we finish this term with a better understanding of the software design and development process.
        \end{abstract}     
    \end{singlespace}
\end{titlepage}

%s\newpage
\pagenumbering{arabic}
\tableofcontents
% 7. uncomment this (if applicable). Consider adding a page break.
%\listoffigures
\clearpage
\listoftables
\clearpage
\begin{flushleft}
% 8. now you write!
\section{Introduction}
Aerolyzer is an application that our group is currently in the processing of developing in collaboration with our clients at NASA JPL. Throughout the past 10 weeks, we have meet weekly with our client for positive discussion, as well as for learning purposes and to formulate a plan for the development of our web application. This document serves to highlight the progress we have made over the past few months, as well as provide a more detailed retrospective of the work we have accomplished thus far. More specifically, we will elaborate on the purpose and overall goals of our project, in addition to focusing on our current status and explaining the progress we have made week-by-week. Lastly, we will provide a thorough discussion on the positives we have encountered this term, what will need to be changed going forward, and the individual actions that we will need to implement  in order for these changes to be made.


\section{Project Purposes and Goals}
The purpose of Aerolyzer is to develop an application capable of analyzing an image and inferring the corresponding aerosol content. To accomplish this, our image detection will uniquely utilize color distribution within the image to identify features necessary for analyzing said aerosol content. Our goal is to create a web application that employs an open source algorithm that uses the location's meteorological conditions, the colors from the image, and analyzed geo-related data to accomplish our desired functionality of providing our user with an accurate representation of the aerosol content in the image they provided.


\section{Current Project Status}
Presently, our team has established a problem statement, developed an in-depth requirements document, researched and chosen the technologies we will be utilizing throughout the development of our web application, and has created a design document which outlines explicitly how we will be using each technology. Moreover, we have created a Github Organization repository for our Aerolyzer team which currently contains PyLint, an integrated version of TravisCI and Coveralls, and a Sphinx Documentation skeleton which will be used for both code and usage documentation.


\section{Weekly Summaries}
The following is a detailed week-by-week summary of our activities, problems, and solutions. The three team member weekly progress reports are condensed into each week. The weeks correspond with our term; weeks 1 and 2 are omitted, as teams had not been assigned or begun working until week 3.
\subsection{Week 3}
This week was a busy one, as we had to get started on the project as a whole. We had our first meeting with our clients: Kim Whitehall and Lewis McGibbney from NASA JPL. After our meeting, Kim sent us some resources via links to look through to help us get started. We looked through these and got a better understanding of our project, which was really helpful for working on the problem statement document. We also determined what the goals of our project would be. 
Since Reilly mentioned her interest in UX, she was given the tentative team role of looking into the UX component of our app and explored some options for displaying photos by looking at other apps with similar functionality. Sophia was assigned research on REST APIs and completed this throughout the week. Jesse has a physics background and therefore was asked to conduct some research for the algorithm implementation, specifically optics and color-analyzing.
Additionally, we completed our problems statement assignment by creating the LaTeX file and Makefile to compile a PDF for Kim and Lewis. This document summarized the expectations for our senior capstone project. They gave us feedback on our draft and approved the final version, which was signed and sent back to us to turn in.
Our Github repo was created this week; our problem statement files were pushed into a new directory. Weekly updates could now be added to the Wiki.

\medskip

As a group, we were not sure whether we are able to make the Github repo public and use an organization or if we are restricted to just private repos. Kim and Lewis contacted the instructor and determined we could use a public repo under an organization.
Another issue we encountered was having difficulty limiting our project to certain main goals. We had a great discussion with our clients to better figure out the scope of our roles. 
Lastly, we had trouble researching articles on the topic at hand (aerosol and atmospheric conditions). The papers regarding the aerosol content were tough to understand at first. 

\subsection{Week 4}
We met with our TA Vee this week for the first time and discussed what to anticipate for our project. He went over some of the expectations for the class and gave us advice on working as a team, interacting with our client, and completing work for the capstone class. The team continued reviewing various documents that the client has been sending to our mailing list regarding aerosol research and sunset image repositories. Each member also continued working on their specific tasks: Reilly looked at UX examples (such as typical photo-centric app layouts) for taking and displaying photos; Sophia completed a mini project with REST; and Jesse became familiar with \texttt{MatPlotLib} and looked up professors at OSU with a background in optics to contact. Additionally, we revised our problem statement based on feedback from lecture and started working on the requirements document with help from our client.

\medskip

Our client helped us figure out how to make our Github profiles public. No outstanding issues were encountered at this time.

\subsection{Week 5}
This week, the team updated and finished our problem statement using feedback from our professors, as well as suggestions from our client. This was a very useful assignment, as definitely have a much better understanding of our project now; everyone on the team is on the same page. A final hard copy was turned in, and the final unsigned version was committed to our repo. We also continued working on our requirements document this week. This was challenging at first, as we were given very little direction in class and found it difficult to find help. However, it was certainly helpful to figure out what exactly we would be doing for the rest of our project. Going through requirements during our meeting with the client was helpful for completing the doc as well.

\medskip

Our client asked us to look at some coding tasks that would be completed at some point before beginning our project. These tasks consisted of working with APIs that allowed us to gather weather data and sunset images from various websites. Additionally, we continued looking at research documents suggested by our client and reviewed our Github workflow document as a team.

\subsection{Week 6}

For this week, we spent the majority of our time finishing up our requirements document. There were some sections that we could not fully complete, as they depended on completing further research regarding the color-analyzing aerosol algorithm, but we just made note of that in the document and will add to it as needed.  The Gantt chart took us quite a bit of time to complete, mainly because we were unsure how to create one. However, we got help from our client during our weekly meeting and were able to complete it based on that information. Together, we wrote out tasks to be completed until Expo based on our functional requirements and converted those into a working Gantt chart. Reilly worked on the first coding task by writing a small Python program that used the Weather Underground API to retrieve meteorological data. The Wunderground API seems pretty useful for our needs, especially considering it is the free version. Our client also provided a script that looked at color channels in a given image that we could use in our software. It was very exciting to see our project form from ideas and now look like the start of an actual application.

\medskip

Finally, we spoke with one of our professors this week to better clarify what problem we are trying to solve with Aerolyzer. She helped us work through finding the best way to convey the goals of our project to others.

\subsection{Week 7}
We made quite a bit of progress this week by finishing up our requirements document and beginning our tech document. Our client gave us several suggestions for improvement on our requirements document and Gantt chart the day that it was due, so we spoke with our TA and were able to turn it in a day later than discussed and committed the doc to Github as well. We did receive some feedback from our TA and will need to make the suggested changes for the final document we put together at the end of the year. 

\medskip

Since there was no school Friday, we met with our client on Wednesday and went over how we would complete our tech document. We came up with 9 different pieces of the Aerolyzer application that we could research to find alternative technologies. These included: uploading photos, getting geo-related data, and documentation for Jesse; the coding environment, image and metadata storage, and extracting EXIF metadata for Reilly; and the web interface, testing, and getting meteorological data for Sophia. Furthermore, our client assigned us each a technology to begin implementing and commit to the repo. Each team member had an issue under the Github repo: Reilly was to add the Pylint config file and supporting Wiki; Jesse, the Sphinx documentation; and TravisCI and Coveralls support for Sophia.

\medskip

One issue we had this week involved our tech review. Since we had already decided on a few of the technologies we would be using as a team and had begun implementing a few of them, our client asked the professor about what was required of the tech review assignment. She did not think we should spend time searching for alternatives when we had already decided to use and begun moving forward with specific technologies. She told us that he said we only had to compare technologies where possible, meaning we did not have to research alternatives for technologies we had already selected.

\subsection{Week 8}
This week, our tech review was committed to github. The chosen technologies for various pieces of our system were Pylint, ExifRead, MySQL, TravisCI and Coveralls, Django, Weather Underground API, MISR, Sphinx, and DropzoneJS. Reilly committed a Pylint file for our project (\texttt{aerolyzer\_lint\_file}) and will be in charge of maintaining this file going forward; she also added our Pylint config information to the wiki. Sophia added our custom .travis.yml file to integrate TravisCI to our code base. Jesse continued working on our Sphinx documentation. Our design doc tex template was committed to Github in lieu of going to class on Thursday. 

\medskip

We ran into a problem checking in our tech review files: somehow \texttt{git push} did not actually push any files and gave a “nothing to commit” message. We ended up adding files manually via the online github interface in order to meet the midnight deadline. Over the next few days we tried to figure out what went wrong, but were not able to find a root issue. However, the problem seems to be fixed now (as we have committed files since then), so hopefully we do not run into this again. We also ran into a problem when implementing the \texttt{.travis.yml} file from Sophia; we were confused about what to add to the content of it. Our clients helped us and Sophia during our team meeting, offering an example .yml file from another project he had worked on. We also ran into the issue of adding Coveralls to the main repo, and not just a personal repo. She fixed this by going into the settings for the main repo to allow coveralls to be integrated. Finally, Reilly had an issue with the Pylint file in her first commit and had added the entire source code directory instead of just the config file. Our client helped us and Reilly remedy this, then closed the issue and merged the pull request.

\subsection{Week 9}
This past week, the team just worked on the design document and completed close to half of it. We also got started on a document for our progress report. Additionally, Sophia worked on fixing the TravisCI .yml file by changing the content based on feedback from our client. We did not meet with our client due to the Thanksgiving holiday.

\medskip

We are confused about our feedback on the tech review. Our client told us that one of our professors said we did not have to compare different technologies for pieces of the system that we had already moved forward with implementing, as that would be an inefficient use of our time. However, a different professor graded our assignment, so we believe there was a miscommunication and we received a poorer grade as a result.

\subsection{Week 10}

This week, we continued working on the design document. We had the design document finished by Thursday, then received the approval of and signature form our client on Thursday night. Only one of our clients signed, though we left a signature space on the document for our other client - our main project manager was on vacation and was not sure whether she would be able to provide a signature before the deadline. The document was turned in Friday morning with our signatures added. We received some feedback from our clients regarding the design doc and will probably end up changing our DBMS. We will discuss this and other changes to the design doc at our last meeting, then make changes to the tech review and design doc accordingly next term. We also met up to discuss dividing up work for the progress report and will now begin working on our assigned sections. We will not have a meeting this Friday due to our client being on vacation, and our last meeting of the term will be this upcoming Friday.

\medskip

Over the weekend and into finals week, we will be writing our progress report and working on the presentation slides. Then we will meet up to practice and record our presentation. This will be our last assignment of the term; everything will be turned in before noon on Wednesday.

\clearpage

\section{Retrospective}
\begin{table}[h!]
\caption{Retrospective of the past 10 weeks}\label{table:1}
\centering
    \begin{tabular}{| p{0.3\linewidth} | p{0.3\linewidth} | p{0.3\linewidth}  |} 
        \hline
        Positives & Deltas & Actions \\ [0.5ex] 
        \hline
        Our team works well together & We need to do a better job of allocating work ahead of time and completing our tasks sooner &  From now on we will allocate task the day an assignment is posted and aim to complete all assignments no later than three days before they are due \\ [0.5ex]
        \hline
        Our team gets along well with our clients, Dr. Whitehall and Dr. McGibbney & We need to do a better job of communicating with our clients consistently throughout the week & To do this we will make better use of our slack channel as questions arise \\ [0.5ex]
        \hline
         Each member of the group brings a unique skill set to the table & We need to ensure that we are utilizing our strengths while also developing our weaknesses & Next term we will discuss in our weekly client meeting how we can implement both of these changes, and proceed from there \\ [0.5ex]
        \hline
        After lengthy discussion with our client, we have determined all the technologies necessary for our web application  & We may need to change some of the technologies we selected. For example, we will likely no longer be using MySQL & We will need to further discuss our technologies with our clients, and determine which technologies may be better suited for our uses \\ [0.5ex]
        \hline
        We have the majority of our initial project development documents completed & According to our TA, we will need to make changes to our Requirements Document and Tech Review next term & We will use the feedback from our TA to flesh out our Requirements Document and Tech Review, as well as provide additional available technologies in the Tech Review for the sections in which we only included one. \\ [0.5ex] 
        \hline
    \end{tabular}
\end{table}

\section{Conclusion}
In summary, over the past 10 weeks our team has made significant progress in establishing a well thought out development plan for our project. Through the creation of our problem statement, requirements document, tech review, and design document, we have manage to create a solid set of guidelines which we can work according to and continually update throughout our development of the web application. Moreover, we not only have a better of understanding of the software design and development process, but we have improved our background knowledge on the science behind our application through continual discussion with our client. To conclude, we are satisfied with the accomplishments we have made over the course of this term, and we are excited for the opportunity to begin developing code in the coming months.

\end{flushleft}
\end{document}